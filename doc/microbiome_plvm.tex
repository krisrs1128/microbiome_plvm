\documentclass[oupdraft]{bio}
% \usepackage[colorlinks=true, urlcolor=citecolor, linkcolor=citecolor, citecolor=citecolor]{hyperref}
\usepackage{url}

% Add history information for the article if required
\history{Received August 1, 2010;
revised October 1, 2010;
accepted for publication November 1, 2010}

\begin{document}

% Title of paper
\title{Latent Variable Modeling for the Microbiome}

\author{
  KRIS SANKARAN$^\ast$, SUSAN HOLMES\\[4pt]
  % Author addresses
  \textit{
    Department of Statistics
    Stanford University
    390 Serra Mall
    Stanford, CA 94305
    United States
  } \\[2pt]
  % E-mail address for correspondence
  {krissankaran@stanford.edu}
}

% Running headers of paper:
\markboth
% First field is the short list of authors
{Sankaran and others}
% Second field is the short title of the paper
{Latent Variable Modeling for the Microbiome}

\maketitle

% Add a footnote for the corresponding author if one has been
% identified in the author list
\footnotetext{To whom correspondence should be addressed.}

\begin{abstract}
  {
    The abstract.
  }
  {
    keyword1
  }
\end{abstract}

\section{Introduction}

Microbiome studies attempt to characterize variation in microbial
abundance profiles across different experimental conditions
\cite{human2012structure}. For example, a study may attempt to describe differences
in microbial commuities between diseased and healthy states or after
deliberately induced perturbations \cite{dethlefsen2011incomplete}.

In the process, there tend to be two complementary difficulties. First, the data
are often high dimensional, measured over several hundreds or thousands of
microbes. Studying patterns at the level of individual microbes is typically
infeasible. Second, it can be important to study microbial abundances in context
of known biological information. For example, it is scientifically meaningful
when a collection of evolutionarily related microbes change in sync with one
another.

\section{Methods}

We now review a few of the statistical modeling techniques that we believe can
be useful building blocks when performing microbiome analysis. Many of these
techniques have been borrowed from text analysis, thinking of the samples by
microbes matrix as a biological analog of the usual document-term matrix. The
idea of transferring these techniques to the microbiome is not new, though its
appropriateness and usefulness has only been explored in relatively limited
settings \cite{shafiei2015biomico, chen2012estimating, holmes2012dirichlet,
  chen2013variable}.

\subsection{Latent Dirichlet Allocation}

Latent Dirichlet Allocation (LDA) is a generalization of multinomial mixture
modeling applicable to count matrices. We adopt the usual topic modeling
terminology, where each document is summarized by a vector of term counts.
Suppose there are $D$ documents across $V$ terms, and that these documents are
assumed mixtures of $K$ underlying topics, where a topic is defined as a
distribution over words.

Let $\theta_{d} \in \simplex^{K}$ represent the $d^{th}$ topic's mixture over the
$K$ underlying topics. Represent the term in the $n_{th}$ word of this document
by $w_{dn}$, and the associated topic by $z_{dn}$. Suppose the $k^{th}$ topic
places weight $\beta_{vk}$ on the $v^{th}$ term, so that $\beta_{\cdot k} \in
\simplex^{V}$. Then, the generative mechanism is

\begin{align*}
w_{dn} \vert \left(\beta_{\cdot k}\right)_{k = 1}^{K}, z_{dn} &\sim \Cat\left(\beta_{\cdot z_{dn}}\right) \\
z_{dn} \vert \theta_{d} &\sim \Cat\left(\theta_{d}\right) \\
\theta_{d} &\sim \Dir\left(\alpha\right) \\
\beta_{\cdot k} &\sim \Dir\left(\gamma\right).
\end{align*}


In the microbiome application, we will find a formulation that marginalizes over
the $z_{dn}$ more convenient. Indeed, a strict analogy between text modeling and
microbiome analysis would consider each living microbe a word $w_{dn}$, while
what we are more interested in are counts of species across samples.
Writing $n_{dv} = \sum_{n = 1}^{N_{d}} \indic{w_{dn} = v}$, we can write the
marginal distribution as

\begin{align*}
n_{d\cdot} \vert \left(\beta_{k}\right)_{1}^{K} &\sim \Mult\left(n_{d\ast}, \beta \theta_{d}\right) \\
\beta_{\cdot k} &\sim \Dir\left(\gamma\right), k = 1, \dots, K \\
\theta_{d} &\sim \Dir\left(\alpha\right), d = 1, \dots D
\end{align*}

\subsection{Dynamic Unigram Model}

While LDA imagines samples being mixtures of fundamental topics on the
$V$-dimensional simplex, the Dynamic Unigram Model identifies a sequence of
samples over time with a curve along this simplex \cite{blei2006dynamic}. To
enforce smoothness in probabilities over over time, a random walk is used is
passed through a softmax function. That is, define,

\begin{align*}
n_{d\cdot} \vert \mu_{t\left(d\right)}  &\sim \Mult\left(\sum_{v} n_{dv}, S\left(\mu_{t\left(d\right)}\right)\right) \\
\mu_{t} \vert \mu_{t - 1} &\sim \Gsn\left(\mu_{t - 1}, \sigma^{2}I_{V}\right) \\
\mu_{0} &\sim \Gsn\left(0, \sigma^{2}I_{v}\right).
\end{align*}
where $S$ is the multilogit link
\begin{align*}
\left[S\left(\mu\right)\right]_{v} = \frac{\exp{mu_{v}}}{\sum_{v^{\prime}} \exp{\mu_{v^{\prime}}}}.
\end{align*}

\subsection{Nonnegative Matrix Factorization}

In LDA, count matrices are modeled by sampling from a multinomials with total
counts coming from the total number of words in each document and probabilities
coming from the rows of $\Theta B^{T}$ where $\Theta = \begin{pmatrix}\theta_{1}
  \\ \vdots \\ \theta_{D} \end{pmatrix} \in \left(\simplex^{K -
  1}\right)_{\downarrow \times D}$ and $B = \begin{pmatrix} \beta_{\cdot 1}
  \dots \beta_{\cdot K} \end{pmatrix} \in \left(\simplex^{V -
  1}\right)_{\rightarrow \times K}$ are $D \times K$ and $V \times K$ matrices
representing document and topic distributions, respectively.


To generalize this idea, it is is possible to model the nonnegative matrix $N$
by the product of low rank matrices, $N \approx \Theta B^{T}$, where now the
only constraints on $\Theta$ and $B$ are that $\Theta \in \reals_{+}^{D \times
  K}$ and $B \in \reals_{+}^{V \times K}$. This is the starting point for a
variety of algorithms in the Nonnegative Matrix Factorization (NMF) literature
\cite{wang2013nonnegative, berry2007algorithms, lee2001algorithms}

Here, we will consider a Gamma-Poisson factorization model (GaP)
\cite{kucukelbir2015automatic, @canny2004gap} which is proposes the hierarchical
model
\begin{align*}
N &\sim \Poi\left(\Theta B^{T}\right) \\
\Theta &\sim \Gam\left(a_{0} 1_{D \times K}, b_{0} 1_{D \times K}\right) \\
B &\sim \Gam\left(c_{0} 1_{V \times K}, d_{0} 1_{V \times K} \right),
\end{align*}
where mean that each entry in these matrices is sampled independently, with
parameters given by the corresponding entry in the parameter matrix.

\section{Simulation Study}
\section{Data Analysis}

\section{Discussion}

\section{Software}

\section{Supplementary Material}

Supplementary material is available online at
\url{http://biostatistics.oxfordjournals.org}.

\section*{Acknowledgments}

{\it Conflict of Interest}: None declared.

\bibliographystyle{biorefs}
\bibliography{refs}

\begin{figure}[!p]
  \centering\includegraphics{figure/betacontours1-1}
  \caption{Different inferenece algorithms for LDA produce different uncertainty
    assessments in small sample sizes, but become comparable as more data arrives.}
  \label{fig:lda_contours}
\end{figure}

\begin{figure}[!p]
  \centering\includegraphics{figure/visualizezinfthetashist-1}
  \caption{Zero inflation poses problems for NMF, even when accounted for in the likelihood. The deterioration is most dramatic when applying VB.}
  \label{fig:zinf_thetas}
\end{figure}
\end{document}
